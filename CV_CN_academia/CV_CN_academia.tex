\documentclass[10pt, letterpaper]{article}

% Packages:
\usepackage[
    ignoreheadfoot, % set margins without considering header and footer
    top=0.3 cm, % seperation between body and page edge from the top
    bottom=0.3 cm, % seperation between body and page edge from the bottom
    left=1.5 cm, % seperation between body and page edge from the left
    right=1.5 cm, % seperation between body and page edge from the right
    footskip=1.0 cm, % seperation between body and footer
    % showframe % for debugging 
]{geometry} % for adjusting page geometry
\usepackage{titlesec} % for customizing section titles
\usepackage{ctex}
\usepackage{tabularx} % for making tables with fixed width columns
\usepackage{array} % tabularx requires this
\usepackage[dvipsnames]{xcolor} % for coloring text
\definecolor{primaryColor}{RGB}{0, 0, 0} % define primary color
\usepackage{enumitem} % for customizing lists
\usepackage{fontawesome5} % for using icons
\usepackage{amsmath} % for math
\usepackage[
    pdftitle={Adrian's CV},
    pdfauthor={Adrian},
    pdfcreator={LaTeX with RenderCV},
    colorlinks=true,
    urlcolor=primaryColor
]{hyperref} % for links, metadata and bookmarks
\usepackage[pscoord]{eso-pic} % for floating text on the page
\usepackage{calc} % for calculating lengths
\usepackage{bookmark} % for bookmarks
\usepackage{lastpage} % for getting the total number of pages
\usepackage{changepage} % for one column entries (adjustwidth environment)
\usepackage{paracol} % for two and three column entries
\usepackage{ifthen} % for conditional statements
\usepackage{needspace} % for avoiding page brake right after the section title
\usepackage{iftex} % check if engine is pdflatex, xetex or luatex

% Ensure that generate pdf is machine readable/ATS parsable:
\ifPDFTeX
    \input{glyphtounicode}
    \pdfgentounicode=1
    \usepackage[T1]{fontenc}
    \usepackage[utf8]{inputenc}
    \usepackage{lmodern}
\fi

\usepackage{charter}

\usepackage{fontspec}
\setmainfont{Times New Roman}

% Some settings:
\raggedright
\AtBeginEnvironment{adjustwidth}{\partopsep0pt} % remove space before adjustwidth environment
\pagestyle{empty} % no header or footer
\setcounter{secnumdepth}{0} % no section numbering
\setlength{\parindent}{0pt} % no indentation
\setlength{\topskip}{0pt} % no top skip
\setlength{\columnsep}{0.15cm} % set column seperation
\pagenumbering{gobble} % no page numbering

\titleformat{\section}{\needspace{4\baselineskip}\bfseries\large}{}{0pt}{}[\vspace{1pt}\titlerule]

\titlespacing{\section}{0pt}{8pt}{3pt} % left margin, space before, space after

\renewcommand\labelitemi{$\vcenter{\hbox{\small$\bullet$}}$} % custom bullet points
\newenvironment{highlights}{
    \begin{itemize}[
        topsep=0.10 cm,
        parsep=0.10 cm,
        partopsep=0pt,
        itemsep=0pt,
        leftmargin=0 cm + 10pt
    ]
}{
    \end{itemize}
} % new environment for highlights


\newenvironment{highlightsforbulletentries}{
    \begin{itemize}[
        topsep=0.10 cm,
        parsep=0.10 cm,
        partopsep=0pt,
        itemsep=0pt,
        leftmargin=10pt
    ]
}{
    \end{itemize}
} % new environment for highlights for bullet entries

\newenvironment{onecolentry}{
    \begin{adjustwidth}{
        0 cm + 0.00001 cm
    }{
        0 cm + 0.00001 cm
    }
}{
    \end{adjustwidth}
} % new environment for one column entries

\newenvironment{twocolentry}[2][]{
    \onecolentry
    \def\secondColumn{#2}
    \setcolumnwidth{\fill, 2.5 cm}
    \begin{paracol}{2}
}{
    \switchcolumn \raggedleft \secondColumn
    \end{paracol}
    \endonecolentry
} % new environment for two column entries

\newenvironment{threecolentry}[3][]{
    \onecolentry
    \def\thirdColumn{#3}
    \setcolumnwidth{, \fill, 4.5 cm}
    \begin{paracol}{3}
    {\raggedright #2} \switchcolumn
}{
    \switchcolumn \raggedleft \thirdColumn
    \end{paracol}
    \endonecolentry
} % new environment for three column entries

\newenvironment{header}{
    \setlength{\topsep}{0pt}\par\kern\topsep\centering\linespread{1.5}
}{
    \par\kern\topsep
} % new environment for the header

\newcommand{\placelastupdatedtext}{% \placetextbox{<horizontal pos>}{<vertical pos>}{<stuff>}
  \AddToShipoutPictureFG*{% Add <stuff> to current page foreground
    \put(
        \LenToUnit{\paperwidth-2 cm-0 cm+0.05cm},
        \LenToUnit{\paperheight-1.0 cm}
    ){\vtop{{\null}\makebox[0pt][c]{
        \small\color{gray}\textit{Last updated in September 2024}\hspace{\widthof{Last updated in September 2024}}
    }}}%
  }%
}%

% save the original href command in a new command:
\let\hrefWithoutArrow\href

% new command for external links:


\begin{document}
    \newcommand{\AND}{\unskip
        \cleaders\copy\ANDbox\hskip\wd\ANDbox
        \ignorespaces
    }
    \newsavebox\ANDbox
    \sbox\ANDbox{$|$}

    \begin{header}
        \fontsize{25 pt}{25 pt}\selectfont \heiti\bfseries 孙宇飞

        \vspace{5 pt}

        \normalsize
        \kern 5.0 pt%
        \mbox{\faEnvelope\ \hrefWithoutArrow{mailto:sunyf20@mails.tsinghua.edu.cn}{sunyf20@mails.tsinghua.edu.cn}}%
        \kern 5.0 pt%
        \AND%
        \kern 5.0 pt%
        \mbox{\faPhone\ \hrefWithoutArrow{tel:+86 188 1310 3532}{188 1310 3532}}%
        \kern 5.0 pt%
        \AND%
        \kern 5.0 pt%
        \mbox{\faGlobe\ \hrefWithoutArrow{https://adriansun.drhuyue.site/}{adriansun.drhuyue.site}}%
        \kern 5.0 pt%
        \AND%
        \kern 5.0 pt%
        \mbox{\faLinkedin\ \hrefWithoutArrow{https://www.linkedin.com/in/sunyufei/}{sunyufei}}%
        \kern 5.0 pt%
        \AND%
        \kern 5.0 pt%
        \mbox{\faGithub\ \hrefWithoutArrow{https://github.com/syfyufei}{syfyufei}}%
    \end{header}

    \vspace{5 pt - 0.3 cm}

    \section{教育经历}



        
        \begin{twocolentry}{
            2020.09–今
        }
        \textbf{清华大学\ 政治学系(大数据政治学方向) \ 博士候选人\ 专业排名: 1/89}\end{twocolentry}

        \vspace{0.01 cm}  % 减少间距
        \begin{onecolentry}
            \begin{itemize}[
                topsep=0.005cm,   % 进一步减少顶部间距
                parsep=0.005cm,   % 进一步减少段落间距
                partopsep=0pt,    % 移除部分顶部间距
                itemsep=0pt,      % 移除项目间距
                leftmargin=0cm + 10pt
            ]
                研究生国家奖学金 \
                清华大学一等奖学金 \
                清华大学国际学生工作奖学金
            \end{itemize}
        \end{onecolentry}

        \begin{twocolentry}{
            2016.09–2020.06
        }
        \textbf{山东大学\ 政管学院 国际政治专业\ 法学学士 \  专业排名: 1/23}\end{twocolentry}

        \vspace{0.01 cm}  % 减少间距
        \begin{onecolentry}
            \begin{itemize}[
                topsep=0.005cm,   % 进一步减少顶部间距
                parsep=0.005cm,   % 进一步减少段落间距
                partopsep=0pt,    % 移除部分顶部间距
                itemsep=0pt,      % 移除项目间距
                leftmargin=0cm + 10pt
            ]
                本科生国家奖学金(2次)\
                山东省优秀毕业生 \
                山东大学一等奖学金
            \end{itemize}
        \end{onecolentry}


    \section{专业职位}

        \begin{twocolentry}{
            2024.08–2025.03
        }
        \textbf{东京大学\ 社会科学研究所\ 客座研究员}\end{twocolentry}

        \vspace{0.01 cm}  % 减少间距
        \begin{onecolentry}
            \begin{itemize}[
                topsep=0.005cm,   % 进一步减少顶部间距
                parsep=0.005cm,   % 进一步减少段落间距
                partopsep=0pt,    % 移除部分顶部间距
                itemsep=0pt,      % 移除项目间距
                leftmargin=0cm + 10pt
            ]
                主讲课程:LLM Prompt Engineering Specialization (英文授课)
            \end{itemize}
        \end{onecolentry}
        
        \begin{twocolentry}{
            2021.10–今
        }
        \textbf{GitHub Campus Experts (校园专家)}\end{twocolentry}

        \vspace{0.01 cm}  % 减少间距
        \begin{onecolentry}
            \begin{itemize}[
                topsep=0.005cm,   % 进一步减少顶部间距
                parsep=0.005cm,   % 进一步减少段落间距
                partopsep=0pt,    % 移除部分顶部间距
                itemsep=0pt,      % 移除项目间距
                leftmargin=0cm + 10pt
            ]
                建立Github清华开发者社群,
                推广“计算平权”开发者多元化项目
            \end{itemize}
        \end{onecolentry}

        \begin{twocolentry}{
            2020.09–今
        }
        \textbf{清华大学R语言工作坊 \ 学生创办人}\end{twocolentry}

        \vspace{0.01 cm}  % 减少间距
        \begin{onecolentry}
            \begin{itemize}[
                topsep=0.005cm,   % 进一步减少顶部间距
                parsep=0.005cm,   % 进一步减少段落间距
                partopsep=0pt,    % 移除部分顶部间距
                itemsep=0pt,      % 移除项目间距
                leftmargin=0cm + 10pt
            ]

                主讲课程:R语言基础; 自然语言处理;
                
                连续5年获得清华大学教改项目支持,选课人数超千人
            \end{itemize}
        \end{onecolentry}

        \begin{twocolentry}{
            2022.07–2022.08
        }
        \textbf{鄂尔多斯市政府金融办公室 \ 主任助理(挂职)}
    
        \end{twocolentry}
    
    \section{学术发表}

        \begin{twocolentry}{
            SSCI Q1

            影响因子: 3.7
            
            通讯作者
        }
        Hu, Yue, \textbf{Yufei Sun} , and Donald Lien. 2022. “The Resistance and Resilience of National Image Building: An Empirical Analysis of Confucius Institute Closures in the USA.” \textit{The Chinese Journal of International Politics} 15(2): 209–26.
        \end{twocolentry}

        \begin{twocolentry}{
            CSSCI
            
            第一作者
        }
        \textbf{孙宇飞}, 杨雪冬: 地方治理的“能力空间”与疫情防控压力的传导变异 ——基于2021年春节期间地方疫情防控政策的实证研究 《广州大学学报(社会科学版)》2023(02 vo 22): 71-81.
    \end{twocolentry}

    \begin{twocolentry}{
        CSSCI
            
        学生第一作者
    }
    胡悦,\textbf{孙宇飞},陈子怡. 扣动公心之弦: 公共服务动机激发机制适用性与稳定性研究. 《公共管理与政策评论》, 2025, 14(1): 54-.
    \end{twocolentry}

    \begin{twocolentry}{
        CSSCI
        
        通讯作者
    }
    李振,\textbf{孙宇飞}: 为何需要助推型政策:理解居民健康意识和行为的不同步 《公共管理与政策评论》, 2021(01 vo 10): 31-41.
    \end{twocolentry}

    \begin{twocolentry}{
        CSSCI
    }
    李振,王浩瑜,\textbf{孙宇飞}等: “条块并举”发包制下的基层治理——以T区乡镇政府的精准扶贫工作为例 《公共行政评论》,2020,13(03):102-117+196-197.
    \end{twocolentry}

    \begin{twocolentry}{
        北大核心
    }
    赵泽群, \textbf{孙宇飞}, 和 邱懿: 高职专业选择满意度影响因素、问题与展望——学生个体视角 《中国职业技术教育》 2021(19): 82–86.
    \end{twocolentry}

    \section{软件开发}

        \begin{twocolentry}{
            独立开发者

        }
        \textbf{\texttt{SurveyAgent}}:
        基于LLM Agent的复杂系统问卷实验仿真工具箱

        \begin{itemize}[
            topsep=0.005cm,   % 进一步减少顶部间距
            parsep=0.005cm,   % 进一步减少段落间距
            partopsep=0pt,    % 移除部分顶部间距
            itemsep=0pt,      % 移除项目间距
            leftmargin=0cm + 10pt
        ]
        \item 使用场景:舆情模拟与干预测试;认知与行为科学实验;治理仿真

        \item 核心功能:LLM Agent客制化生成;问卷自动构造与动态适配;用户行为模拟与干预实验

        \end{itemize}
        \end{twocolentry}

        \begin{twocolentry}{
            独立开发者
        }
        \textbf{\texttt{Bureaucratese}}:
        复杂文本中官方话语风格(“官腔”)的规模化识别与提取工具,下载量1k+

        \begin{itemize}[
            topsep=0.005cm,   % 进一步减少顶部间距
            parsep=0.005cm,   % 进一步减少段落间距
            partopsep=0pt,    % 移除部分顶部间距
            itemsep=0pt,      % 移除项目间距
            leftmargin=0cm + 10pt
        ]
        \item 使用场景:舆情分析、异常文本监测、政策文本分析和监测
        \item 核心功能:基于\texttt{BERT}的语义层级识别;即插即用 API 服务;可扩展词典管理

        \end{itemize}
        \end{twocolentry}

        \begin{twocolentry}{
            开发负责人
        }
        \textbf{\texttt{Regioncode}}:中国行政区划名称对齐工具, 下载量12k+
        
        \begin{itemize}[
            topsep=0.005cm,   % 进一步减少顶部间距
            parsep=0.005cm,   % 进一步减少段落间距
            partopsep=0pt,    % 移除部分顶部间距
            itemsep=0pt,      % 移除项目间距
            leftmargin=0cm + 10pt
        ]
        \item 使用场景:清洗跨年行政区划字段,实现名称、地理代码、拼音等信息的互转,提升数据合并容错度

        \item 核心功能:多维字段互转;历史区划追踪;模糊名称识别

        \end{itemize}
        
    \end{twocolentry}



    
    \section{语言与技术}

        \begin{onecolentry}
            \textbf{外语能力:} 英文 IELTS: 7 \hspace{5em}
            \textbf{编程语言:} R, SQL, Python \hspace{5em}
            \textbf{工具使用:} Git, \LaTeX, Quarto, Trae
        \end{onecolentry}

        \begin{onecolentry}
            \textbf{跨文化能力:} 独旅超过30个国家,在海外有长期生活经历\hspace{5em}
        \end{onecolentry}

\end{document}